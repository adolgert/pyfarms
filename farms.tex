\documentclass{article}

\usepackage{times}
\usepackage{mathptmx}
\usepackage[margin=1in]{geometry}
\usepackage{graphicx}
\usepackage{hyperref}

\title{Farm Infection Implementation}
\author{Andrew J.\ Dolgert}
\date{5 July 2015}

\begin{document}
\maketitle

Let's look at direct contact spread for \textsc{naadsm}~3.2,
as specified in the \textsc{naadsm} Model Specification 1.2.1.
Start with Sec.~A4.1 Direct contact spread, page A--10.

In 1, first detection of the disease probably refers not to
detection by a property owner but to first reporting of
disease, as discussed later in the document. The multiplier, $k$,
scales movement rate, so it must start at one and decrease to zero
on some deterministic scale.

In 2, we start with a possible source of infection. Every source
creates the same number of shipments, or the same distribution
of shipments, depending on whether shipment numbers are stochastic.
There are $k$ shipments per day, or \emph{Poisson(k)} shipments per day.
This seems to indicate that, even during an outbreak, a farm produces
the same number of shipments, but we will see later that shipments
are dropped in the algorithm, indicating that the intention is to
model a farm which decreases its shipments under some circumstances.

Only those units which are not quarantined or destroyed are
put into a list. All reported units are automatically quarantined,
so the only way a shipment goes to an already infected unit is if it wasn't
reported. Of course there are no shipments if there are no destinations;
that's a special case. Making a list of only those places that
can receive goods means that more shipments than normal will be pushed to
remaining susceptible farms.

Sampling to find a destination farm does two interesting things.
First it samples only from the list of possible farms.
Second it chooses a distance, which is a radius from the farm,
and selects among farms closest to that distance. If there were
five farms at these radii , $(2, 2.9, 3, 3.1, 4)\:\mbox{km}$,
then the farm $3\:\mbox{km}$ away is much less likely to receive
a shipment. It's an accidentally biased kernel. If the distance
kernel is $m(r)$ and the \emph{available\/} farms are labeled
in order with distances $r_j$, then the probability of choosing
a particular one is
\begin{equation}
  p_j=\int_{(r_{j-1}+r_j)/2}^{(r_{j+1}+r_j)/2} m(r)dr.\ref{eqn:farmprob}
\end{equation}
There is a mention
of farms the ``same'' distance from each other. This sameness means
they are within, according to the code, 0.01 apart. I'm guessing
this is in kilometers. These same-distance farms complicate
calculations because, given several same-distance farms,
ones forbidden by zone are excluded, which means ones allowed
by zone, within a certain distance, are included.

At this point, shipments forbidden by zone movement are dropped.
Up until now, the number of shipments was affected only by the 
movement multiplier, $k$. For zone movement, shipments will be dropped
unless there is a same-distance farm not excluded by zone rules,
in which case the $p_j$ for the same-distance farms are increased.
The percentage dropped is a reduction to the overall number
of shipments to leave. This percentage comes from the likelihood
of choosing a zone-forbidden farm from the list of otherwise-available
farms.
Forbidding transfer by zone reduces the overall shipments per day
by a factor determined from the probability of having arrived
at any zone-forbidden farms, calculated with Eq.~\ref{eqn:farmprob}.
This rescales the multiplier by $1-\sum_n p_n$ where $n$ runs over
farms forbidden by zone rules. The new probability for choosing
a non-forbidden farm is $p_j/\sum_j p_j$ which excludes the forbidden
ones.

\end{document}

